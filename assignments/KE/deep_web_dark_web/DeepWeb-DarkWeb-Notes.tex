\documentclass[12pt]{article}

\usepackage{bm}
\usepackage[top=1in, bottom=1.25in, left=1.25in, right=1.25in]{geometry}

\title{Essay on Deep Web and Dark Web}
\date{2020-10-12}
\author{Bibek Pandey}

\begin{document}

\begin{titlepage}
   \begin{center}
       \vspace*{1cm}

       {\huge An Essay on Deep Web and Dark Web} \\
       \vspace{0.5cm}
       Assignment Notes

       \vfill

       By \\
       \vspace{0.5cm}
       Bibek Pandey \\
       \textbf{076mscsk003}
       \vspace{1.5cm}

       \vfill

       An assignment presented to \\
       \vspace{0.3cm}
       \textbf{Dr.\ Aman Shakya}

       \vspace{1.5cm}
       \textbf{2020 October 12}

   \end{center}
\end{titlepage}

\section*{Introduction}
With the advent of World Wide Web, the era we've been living has transformed to
what we call "The age of Information". Although the web had a few documents
when it started, as the internet took over the world, more and more information
were added. We interchangebly use the words "web" and "internet"
however there is a subtle difference: internet is the network of devices around
the world while web is a way to access the infromation in that network,
typically using HTTP protocol.

And for some decades, after the boom of information in the internet, the web
has been augmented by the search engines like yahoo, bing, google and
duckduckgo. We query information in the search engines and they magically
present us with the link to relevant web pages. How do they do this? The search
engines have what are called crawlers which "crawl" through the web, hopping
from one page to another through the hyperlinks and index the web pages. A
crawler is a program whose sole task is to follow links and gather information.
It seems that the search engines have information/location about all the web
pages.
\newline But do they?
\newline It is astonishing to know that internet also consists of
"Deep Web" and "Dark web" which comprise about 96\% of the information in the
internet. The regular web that we see shares just 4\% of information.

\section*{The Deep Web}
Deep web referes to the part of the world wide web which has web pages which
are not pointed by any links and thus is not indexed. That is, crawlers cannot
reach or find them due to which search engines cannot include them in their
results. The users can access a page in deep web through exact link.
Almost every time we search internally on a website, we’re accessing deep web
content. An example of accessing the deep web would be we accessing our emails
and other account information which obviously search engines don't have access to.
Similarly, when we are in a public library's website and we browse books there,
that's deep web content because normal search engines can't index the catalog
of such libraries as well. In general, the deep web consists of the following
information which people do not want to turn up in a web search:
\begin{itemize}
    \item [-] The content of our personal email accounts
    \item [-] The content of our social media accounts
    \item [-] The content of our online banking accounts
    \item [-] Data that companies store on their private databases
    \item [-] Content contained within scientific and academic databases
    \item [-] Medical records
    \item [-] Legal documents
\end{itemize}

\section*{The Dark Web}
This also cannot be accessed by search engines and hence do not appear in
search results. But the contents in dark web range from information to unhuman
criminal activities. Unlike deep web it can contain very harmful, sensitive and
disturbing content. One major difference with deep web is that, dark web cannot
be accessed using regular web browsers. The special browsers are called `Tor
browsers', TOR standing for `The onion router'. TOR is a hidden service
protocol which lets users have anonymous surfing through dark web. Dark
websites are completely different than normal websites and have servers that
stay inside the TOR network. The websites have `.onion' at their end unlike
`.com', `.gov', etc. The following things available in dark web:
\begin{itemize}
    \item [-] Academic papers, books which might not be available elsewhere
    \item [-] Stolen information
    \item [-] Dangerous Malwares which can have sensitive accesses like our webcams/microphone.
    \item [-] Illegal drugs
    \item [-] Illegal and disturbing services and content like hit-man, child pornography, etc.
\end{itemize}

\section*{}
Although the deep web is very safe and harmless unlike the dark web, it's very
necessary to safely navigate both. In the context of deep web, it means using
strong passwords for private accounts and downloading from trusted sites only.
In dark web, we should avoid even visiting suspicious sites. Most of the
transactions around the dark web take place using bitcoin which is very
anonymous, but people are very likely to be scammed.

To conclude, the web is a great source of knowledge and information. It just
has many things, relevant and irrelevant, safe and harmful, good and bad. It's
upto the user to be aware choose what to visit.

\section*{References}
\begin{itemize}
    \item[-] \textit{https://us.norton.com/internetsecurity-how-to-how-can-i-access-the-deep-web.html}
    \item[-] \textit{https://heimdalsecurity.com/blog/deep-web-vs-dark-web-what-is-each/}
\end{itemize}

\end{document}
